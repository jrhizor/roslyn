
\documentclass[11pt]{report}

\usepackage{epsfig}

\usepackage{USCthesis2004}

% Yu-Fen's changes to the defaults
  % Titles of sections:
  %       \renewcommand{\refname}{REFERENCES}
          \renewcommand{\bibname}{BIBLIOGRAPHY}
  % Fonts for these titles
        \renewcommand\contentsname{\centerline{\bf Table of Contents}}
        \renewcommand\listfigurename{\centerline{\bf List of Figures}}
        \renewcommand\listtablename{\centerline{\bf List of Tables}}


%new environments

\newtheorem{definition}{Definition}

\newtheorem{theorem}{Theorem}

\newtheorem{proposition}[theorem]{Proposition}

\newtheorem{lemma}[theorem]{Lemma}

\newtheorem{corollary}[theorem]{Corollary}

\newtheorem{remark}{Remark}



% roman numbering

% remove the next two lines for arabic numbering

\renewcommand{\thechapter}{\Roman{chapter}}

\renewcommand{\theenumi}{(\roman{enumi})}



\title{Roslyn: The Tour Guide Robot}



\author{Jared Rhizor, Timothy Sweet, and Nishok Yadav}



\date{May 2014}



% You don't need the next line if the copyright year is the current year

\renewcommand{\copyrightyear}{2014}


% You can adjust the line spacing by redefining \stretchratio.

% For example  \renewcommand{\stretchratio}{1}

% will typeset the document in single spacing.
\renewcommand{\stretchratio}{2}



\ProcessOptions
\begin{document}



\maketitle



\begin{preface}



\pagebreak   % Dediciation


\pagebreak   % Acknowledgements

\mbox{}



\vspace{0.5in}

\begin{center}

{\Large \bf Acknowledgements}

\end{center}

The authors would like to thank of number of people who made this project possible. First, our advisor, Dr. Dave Feil-Seifer, whose enthusiasm and support drove this project to success. Second, the University of Nevada Robotics Research Lab, including its directors Dr. Dave Feil-Seifer (again) and Dr. Monica Nicolescu for providing us a place to work and equipment to use. Third, to Dr. Sergiu Dascalu for providing guidance on the project as a whole, and especially in developing all of the specifications, diagrams, etc. in this document. Fourth, to the University of Nevada, Reno Honors Program and Office for Interdisciplinary and Undergraduate Research for providing funding for Roslyn’s additional hardware. Fifth, to Team 4 (Luke, Jessie, Jake, and Blake) from our 2014 CS 426 Senior Projects course, who helped at every step of development, contributing their time, expertise, food, and study partners. Sixth, to the Open Source Robotics Foundation, Willow Garage, and the entire Robot Operating System community for doing most of the heavy lifting in developing the robotic control and interface components.

%Uncomment next line if you want to add acknowledgements to content

%\addcontentsline{toc}{chapter}{Acknowledgements}





\begin{singlespace}

 \tableofcontents   % content

 \listoffigures     % list of figures

\end{singlespace}



\pagebreak

\begin{abstract}

The tour guide robot Roslyn is an autonomous system that can provide tours to anybody on the second floor of the Scrugham Engineering and Mines building. The robot is not only able to navigate the halls, but navigate them in a socially acceptable manner. This means it does not collide with pedestrians while in motion and takes walking speed into consideration. In order to interact with the user, there is a touch-screen display mounted on top of the moving robot base. As the robot moves throughout the hall, information appears on the screen related to the landmarks being passed (i.e. laboratories and offices). The user is also able to select a destination on the screen in order to begin the tour.

\end{abstract}

\addcontentsline{toc}{chapter}{Abstract}



\end{preface}


\chapter{First Chapter}

The tour guide robot Roslyn is a socially aware autonomous system that can lead tours of the Computer Science and Engineering Department at the University of Nevada, Reno. This requires socially acceptable navigation within the second floor of the Scrugham Engineering and Mines building.
Tours of the Computer Science and Engineering Department are currently given by faculty and staff, or student ambassadors (typically from other departments). This creates some difficulties for visitors:
\begin{itemize}
 \item A tour can only be scheduled when a faculty or staff member is available. This may conflict with class times, office hours, or other responsibilities
 \item Giving tours distracts faculty and staff from their normal studies, teaching, research, etc. especially with large groups
 \item Tours given by student ambassadors are generally limited to superficial knowledge of the department, such as the location of the department office. A typical student ambassador would not be able to answer specific questions about a program, lab, or other facility. 
\end{itemize}

A robotic tour guide could address many of these issues:
\begin{itemize}
 \item A tour with a robot could be scheduled for any time the building is accessible, even after normal working hours. Tours could be scheduled in real-time without concerns about a faculty or staff member’s conflicting schedule. Tours could also be given on demand for visitors on site without an appointment
 \item A robot could be dedicated entirely to giving tours, and would not become distracted by other activities
 \item A robot could be provided up-to-date information by a departmental authority, or lookup information in real time.
\end{itemize}

Some challenges are present for a robotic tour guide in a situated academic environment, all of which shall be addressed in this thesis:
\begin{itemize}
 \item Safe navigation: both globally and locally. A global plan must be generated for every destination which will optimize the route taken to the destination. A local plan must also be generated which avoids obstacles such as walls, items in a hallway, and moving people. This local plan must also lead the user on a reasonable path and a reasonable pace.
 \item Self safety: the robot must not endanger itself. Some potential safety concerns include stairwells and objects mounted outside of the robot’s view (ie drinking fountains, which are mounted above a laser range finder’s plane of view).
 \item Power and portability constraints: the wireless network installed in the Scrugham Engineering and Mines building is insufficient for real-time sensor data transmission, thus all processing must be performed on the robot. Thus there are constraints on the amount of power the robot can deliver to its computational components, the battery life of the robot and these components, and the maximum reasonable size of these components.
\end{itemize}


In order for the robot to be socially aware, it must not only lead its user at a comfortable distance and pace, but also keep other humans and obstacles at a safe distance. A human tour guide would not walk exceptionally fast away from their group or walk uncomfortably close to them, thus the robot should not either.

The tour guide robot is primarily composed of three distinct nodes, which are each discussed independently due to their complexity. The first node, labeled ROS Nodes in Fig. 1, is a collection of pre-existing open source ROS nodes which provide data and processing facilities to the system. These nodes are not being developed by the team, but are distributed by the Open Source Robotics Foundation.

The second node, labeled User Interface in Fig. 1, provides a user-friendly interaction between the robot and user(s). This allows the users to select a destination, as well as request and view more information about their surroundings via a touchscreen tablet on the robot.

The third node, labeled Social Navigation in Fig. 1, safely navigates the robot through the dynamic environment and ensures proper social interactions with the user.

Roslyn is composed of two independent programs:
Roslyn is able to map an environment and use the map for future navigation. This is done the first time Roslyn is introduced to a particular building and does not have to be repeated.
Roslyn uses the building map in its second program along with its sensors and communication pathways to communicate with the user interface and drive itself.
Roslyn is composed of a number of hardware components:
The robotic base: a Pioneer 3DX
An internal computer: a Raspberry Pi Model B Revision 2
A laser rangefinder: a SICK LMS 100 or a Hokoyu URG-04LX-UG01
A laptop: Intel® Core™ i5-2450M @ 2.50GHz with 8 GB of RAM or better
The user interface: a Nexus 7 2013
A custom stand for the Nexus 7 2013
The rest of this thesis is organized as follows:...



\section{Title Page}





\section{Preface} The preface section in the sample is typeset by the

following.



\subsection{Spacing and Bibliography}




\bibliographystyle{plain}



\begin{thebibliography}{99}

 \bibitem{USCthesis} \verb=USCthesis2000.sty=,

 \LaTeX\,2\raisebox{-0.2ex}{$\varepsilon$} style file for USC dissertations

 and theses according to the regulations published by the Graduate

 School, Feb 2000.

\end{thebibliography}



\end{document}
